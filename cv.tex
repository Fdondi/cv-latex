%%%%%%%%%%%%%%%%%%%%%%%%%%%%%%%%%%%%%%%%%
% "ModernCV" CV and Cover Letter
% LaTeX Template
% Version 1.1 (9/12/12)
%
% This template has been downloaded from:
% http://www.LaTeXTemplates.com
%
% Original author:
% Xavier Danaux (xdanaux@gmail.com)
%
% License:
% CC BY-NC-SA 3.0 (http://creativecommons.org/licenses/by-nc-sa/3.0/)
%
% Important note:
% This template requires the moderncv.cls and .sty files to be in the same 
% directory as this .tex file. These files provide the resume style and themes 
% used for structuring the document.
%
%%%%%%%%%%%%%%%%%%%%%%%%%%%%%%%%%%%%%%%%%

%----------------------------------------------------------------------------------------
%	PACKAGES AND OTHER DOCUMENT CONFIGURATIONS
%----------------------------------------------------------------------------------------

\documentclass[11pt,a4paper,sans]{moderncv} % Font sizes: 10, 11, or 12; paper sizes: a4paper, letterpaper, a5paper, legalpaper, executivepaper or landscape; font families: sans or roman

\moderncvstyle{classic} % CV theme - options include: 'casual' (default), 'classic', 'oldstyle' and 'banking'
\moderncvcolor{purple} % CV color - options include: 'blue' (default), 'orange', 'green', 'red', 'purple', 'grey' and 'black'
%\usepackage[italian]{babel}
\usepackage[utf8]{inputenc}
\usepackage{lmodern}
\usepackage{enumitem}
\usepackage{fontawesome5}
\usepackage{ragged2e}
\usepackage{graphicx}
\usepackage{wrapfig}
\usepackage{datatool}
\usepackage{paracol}
\usepackage{url}
\usepackage{varwidth}

\definecolor{customTeal}{RGB}{0, 128, 128} % RGB for Teal
\definecolor{customTurquoise}{RGB}{64, 224, 208} % RGB for Turquoise
\definecolor{lightGrey}{RGB}{211, 211, 211} % RGB for Light Grey
\definecolor{lightBlue}{HTML}{79B8CF}
\definecolor{blueGray}{HTML}{748FB0}
\definecolor{lightorange}{RGB}{255, 229, 204}
\definecolor{navyBlue}{RGB}{0, 0, 139}

\definecolor{boldBlue}{RGB}{25, 25, 100} % MidnightBlue
\definecolor{quietBlue}{RGB}{150, 200, 250} % SteelBlue

\definecolor{titleBlue}{HTML}{4A6483}

% Define purple colors
\definecolor{boldPurple}{RGB}{153, 50, 204} % DarkOrchid
\definecolor{quietPurple}{RGB}{175, 145, 200} % Lavender

% Ensure that what displays C#, grabbing the text gets C-<ascii hex-23 = #>
\newcommand*{\Csh}{C\texttt{\#} }

\newcommand{\repeatsymbol}[2]{%
 \ifnum#1>0%
 	\foreach \n in {1,...,#1}{#2}%
 \fi%
}

\newcommand{\skilllevel}[1]{%
	\repeatsymbol{#1}{\faCircle}\repeatsymbol{\numexpr5-#1\relax}{\faCircle[regular]}%
}

\newcommand{\skl}[1]{%
	\textcolor{white}{#1}% Still presents just the number to document text grab/copy
	\textcolor{blueGray}{\skilllevel{#1}}%
}

\usepackage[scale=0.9, top=2cm, bottom=2cm, left=0.7cm, right=0.7cm]{geometry} % Reduce document margins
\setlength{\hintscolumnwidth}{3cm} % Uncomment to change the width of the dates column
%\setlength{\makecvtitlenamewidth}{10cm} % For the 'classic' style, uncomment to adjust the width of the space allocated to your name

%----------------------------------------------------------------------------------------
%	NAME AND CONTACT INFORMATION SECTION
%----------------------------------------------------------------------------------------

\firstname{Francesco} % Your first name
\familyname{Dondi} % Your last name

% All information in this block is optional, comment out any lines you don't need
\title{\small \\[1em] \color{black}{Neueste Version dieses Lebenslaufs bei: \href{https://dondi.fyi/cv_de}{https://dondi.fyi/cv\_de}}}
\email{francesco314@gmail.com}
\mobile{+41 76 456 50 32}
\social[linkedin]{francesco-dondi}
\social[github]{Fdondi}
\address{Zugerstarsse 66, 8810 Horgen, ZH}
\extrainfo{\color{lightgray}\raisebox{3mm}{\rule{5cm}{0.4pt}} \\[-0.9em] Geboren 29.10.1990 \\ Italienischer Staatsbürger, C-Ausweis \\
Verheiratet, keine Kinder}
\photo[80pt][0pt]{me.jpg} % The first bracket is the picture height, the second is the thickness of the frame around the picture (0pt for no frame)

%----------------------------------------------------------------------------------------

% Store the old definition of \makecvtitle
\let\oldmakecvtitle\makecvtitle

% Redefine \makecvtitle based on the old definition
\renewcommand*{\makecvtitle}{%
  \hfil% push content to the right
  \oldmakecvtitle%
}

\newcommand{\ToolList}[1]{
\hfill
\begin{minipage}[t]{0.15\textwidth}
{\tiny
  \begin{itemize}
    \setlength\itemsep{-0.3em}
    #1
  \end{itemize}
}
\end{minipage}
}

\newcommand{\mysection}[1]{%
	\textcolor{color1}{\Large{#1}}%
}

\newcommand{\tsection}[1]{%
	\ & \ \\	
	\mysection{#1} & \\%
}

\newcommand{\tssection}[1]{%
	\textcolor{quietPurple}{\textbf{#1}} & \\%
}

\newcommand{\tskl}[2]{%
	#1 & \skl{#2} \\
} 

\newcommand{\Company}[1]{%
\textcolor{titleBlue}{\textbf{\large #1}}
}

\newcommand{\Position}[1]{
\textcolor{color1}{#1}
}

\newcommand{\Location}[1]{%
\textcolor{quietPurple}{\textbf{#1}}
}

\newcommand{\MyCvEntry}[4]{%
\cventry{#1}{\Position{#2}}{}{\Company{#3}}{}{
\noindent
\begin{varwidth}{0.65\textwidth}
\footnotesize #4
\end{varwidth}
}
}

\newenvironment{compactitemize}
  {\vspace{-0.5em}\begin{itemize}\setlength{\itemsep}{-0.3em}\setlength{\parskip}{0pt}}
  {\end{itemize}}

\begin{document}
\pdfinterwordspaceon

\makecvtitle % Print the CV title
\vspace{-3em}
\columnratio{0.75}
\begin{paracol}{2}
%\section{Wer ich bin}
%Als erfahrener Softwareentwickler beherrsche ich die Entwicklung robuster Softwarelösungen, SQL-Datenbankoptimierung und effektive Datenanalyse. Von der Sicherung von Serverkonfigurationen bis zur Entwicklung komplexer Softwarelösungen bin ich in allen Entwicklungsphasen engagiert. Kollegen schätzen meine Zuverlässigkeit, Intelligenz und Teamfähigkeit.

\section{2023–heute: Ära der KI-Startups}

\MyCvEntry{03/2025 - 05/2025}{Backend Engineer}{Hoshii}{
Geliefert: Ein React-Webshop, integriert mit MongoDB und Go-Backend, inklusive Erweiterung der API. Verbesserung der KI-Orchestrierungspipeline.
}

\MyCvEntry{09/2024 - 01/2025}{Senior Machine Learning Engineer}{Giotto AI}{
Hat zur Spitzenforschung beigetragen, die sich mit der ARC-Herausforderung auseinandersetzt. Besonderer Schwerpunkt lag auf der Datenerweiterung. Nutzung von BART zur Klassifizierung und Pytorch/Lightning für effizientes Multi-GPU-Training.
}

\MyCvEntry{03/2024 - 05/2024}{Software Developer}{Deep Impact}{
Sammelte direkte Erfahrungen mit Computer Vision sowohl auf Cloud-Plattformen als auch auf eingebetteten Edge-Geräten (Nvidia Jetson).
Leistete Beiträge durch die Überarbeitung der Docker-Konfiguration und die Behebung von Softwarefehlern.
}

\section{2021–2023: Kapitel Cloud-Daten­warehousing}
\MyCvEntry{03/2023 - 09/2023}{Softwareentwickler}{DoubleCloud}{
Wartung und Entwicklung der Infrastruktur um Managed ClickHouse, insb.:
\begin{compactitemize}
\item Absicherung des Metrikenservers durch Hinzufügen von Nginx
\item Verbesserung der SaltStack-Konfiguration
\item Fehlerbehebung in Telegraf Go-Plugins
\end{compactitemize}
}

\MyCvEntry{06/2021 - 09/2022}{Software Developer}{Firebolt}{
Wartung und Entwicklung des SQL-Optimierungsmotors und der Testinfrastruktur.
\begin{compactitemize}
\item Erweiterung der unterstützten SQL-Syntax und Optimierungen
\item Anpassung des Testframeworks für neue Testserien
\item Beitrag zur Neuschreibung des Optimierungsmotors
\end{compactitemize}
}

\section{2020–2021: Finanz-Zwischenspiel}
\MyCvEntry{08/2020 - 02/2021}{Developer / Analyst}{F-Trust}{
\begin{compactitemize}
\item Wartung und Entwicklung eines Handelsmotors (C++)
\item Neugestaltung und Migration der Architektur der Datenspeicherung
\item Extraktion und Integration von Daten aus mehreren Log-Streams
\item Datenanalyse zur Hervorhebung von Verbesserungsmöglichkeiten
\end{compactitemize}
}

\section{2017-2020: Google}
\MyCvEntry{10/2017 - 05/2020}{Software Developer}{Google}{
Im Bereich Suche, ein ursprünglich experimentelles Tool in den produktiven Einsatz überführt (C++).
\begin{compactitemize}
\item Verbesserung der Laufzeit von Tagen auf Stunden
\item Unterstützung für neue Datentypen
\item Ergebnisse leicht visualisierbar und überwachbar
\item Beschleunigung und Vereinfachung des Onboarding-Prozesses
\end{compactitemize}
}

\section{2015-2017: Erste Erfahrungen}
\MyCvEntry{04/2016 - 10/2017}{Software Developer}{Ascent Software}{
Wartung und Entwicklung eines Treibers (C++) in der Automobilindustrie, der sowohl auf regulären Computern als auf eingebetteten Systemen (QNX) lief.
}

\MyCvEntry{04/2015 - 04/2016}{Junior Developer}{Rulex Inc./CNR}{
Beschleunigung von C-Algorithmen mit CPU (OpenMP) und GPU (Nvidia CUDA) Parallelisierung. Übertragung von MATLAB-Algorithmen nach Python.
}

\switchcolumn
\begin{tabular}{lc}
\tsection{Skills}
\tssection{Languages}
\tskl{C++}{5}
\tskl{Python}{5}
\tskl{Nvidia CUDA}{5}
\tskl{BASH}{4}
\tskl{Go}{3}
\tskl{Java}{2}
\tskl{Julia}{2}
\tskl{\Csh}{2}
\tssection{Data Science/ML}
\tskl{SQL}{5}
\tskl{- Clickhouse SQL}{5}
\tskl{- PostgreSQL}{4}
\tskl{MongoDB} {3}
\tskl{Pytorch}{4}
\tskl{TensorFlow}{3}
\tskl{Pandas}{4}
\tskl{Numpy}{3}
\tssection{Development}
\tskl{Git}{5}
\tskl{Cursor}{4}
\tskl{Jenkins CI}{3}
\tskl{Gitlab CI}{3}
\tssection{Web}
\tskl{React}{3}
\tskl{REST API}{3}
\tskl{Gin}{3}
\tskl{gRPC}{3}
\tssection{Cloud}
\tskl{Docker}{4}
\tskl{Terraform}{2}
\tskl{Helm}{1}
\tskl{Linux}{4}
\tskl{AWS}{3}
\tskl{GCloud}{3}
\tskl{Windows}{3}
\tsection{Languages}
\tskl{English}{5}
\tskl{Italiano}{5}
\tskl{Deutsch}{3}
\tskl{Français}{3}
\end{tabular}

\end{paracol}

\pagebreak

\section{Ausbildung}

\cventry{02/2024--10/2024\hfill}{\href{https://sce.ethz.ch/en/programmes-and-courses/search-current-courses/cas/cas-eth-ml-fin-ins.html}{CAS ETH in Machine Learning in Finance and Insurance}}{ETH Zürich}{}{}{}
\cventry{01/2023--03/2024 \hfill}{Deutschkurs A2-B2}{ECAP Zürich}{}{}{}
\cventry{01/2023--06/2023 \hfill}{\href{https://execed-online.imperial.ac.uk/business-analytics}{Business Analytics: From Data to Decisions}}{Imperial College Business School}{}{}{}
\cventry{2009--2015\hfill}{\href{http://www.unipd-scuolagalileiana.it/en/}{Galileian School of Higher Education}}{University of Padua}{Bereicherungsprogramm}{\textit{98/100}}{}
\cventry{2012--2014\hfill}{Master Degree in Mathematics}{University of Padua}{}{\textit{108/110}}{}
\cventry{2009--2012\hfill}{Bachelor Degree in Mathematics}{University of Padua}{}{\textit{110/110 cum Laude}}{}
\cventry{2004--2009\hfill}{High School - Science Track, added IT, French}{Liceo Fermi}{Bologna}{\textit{100/100 cum Laude}}{}

\newcommand{\BeginCourses}{%
	\begin{tabular}{r@{\hspace{1em}}c@{\hspace{1em}}p{0.5\textwidth}}
}
\newcommand{\EndCourses}{\end{tabular}}

\newcommand{\Course}[3]{%
\hspace{1.5em} #1 & \textit{#3} & \textbf{#2} \\
}

\begin{paracol}{2}
\section{Weiterbildung}
\switchcolumn
\vspace{1.5em}
\mysection{Skills}
\end{paracol}

\subsection{Datenwissenschaft/Analytik}
\begin{paracol}{2}
\switchcolumn[0]*
\BeginCourses
\Course{Nov 2023}{Data Fluency: Exploring and Describing Data}{Linkedin}
\Course{Nov 2023}{Power BI Essential Training}{Linkedin}
\Course{Oct 2023}{Business Intelligence for Consultants}{Linkedin}
\Course{Oct 2023}{Data Analytics for Business Professionals}{Linkedin}
\Course{Oct 2023}{Excel: Economic Analysis and Data Analytics}{Linkedin}
\Course{Oct 2023}{The Non-Technical Skills of Effective Data Scientists}{Linkedin}
\EndCourses
\switchcolumn
\begin{tabular}{p{3cm}c}
\tskl{Data Analysis}{3}
\tskl{Data Visualization}{4}
\tskl{Probability}{4}
\tskl{Microsoft Power BI}{2}
\tskl{Microsoft Excel}{5}
\tskl{Pandas}{4}
\end{tabular}
\end{paracol}

\subsection{Maschinelles Lernen/LLMs/AI}
\begin{paracol}{2}
\switchcolumn[0]*
\BeginCourses
\Course{Jan 2024}{Generative AI with Large Language Models}{Coursera}
\EndCourses
\switchcolumn
\begin{tabular}{p{3cm}c}
\tskl{LLMs}{2}
\end{tabular}
\end{paracol}

\subsection{Programmiersprachen}
\begin{paracol}{2}
\BeginCourses
\Course{Nov 2023}{\Csh Essential Training 1: Types and Control Flow}{Linkedin}
\Course{Oct 2023}{Learning MATLAB}{Linkedin}
\EndCourses
\switchcolumn
\begin{tabular}{p{3cm}c}
\tskl{\Csh}{2}
\tskl{MATLAB}{3}
\end{tabular}
\end{paracol}

\subsection{Embedded-Entwicklung}
\begin{paracol}{2}
\BeginCourses
\Course{Nov 2023}{C Programming for Embedded Applications}{Linkedin}
\EndCourses
\switchcolumn
\begin{tabular}{p{3cm}c}
\tskl{C Embedded}{3}
\end{tabular}
\end{paracol}
\subsection{Cloud und DevOps}
\begin{paracol}{2}
\BeginCourses
\Course{Feb 2023}{AWS Certified Solutions Architect - Associate (SAA-C02) Cert Prep: 1 Cloud Services Overview}{Linkedin}
\Course{Jan 2023}{DevOps Foundations: Going Cloud Native}{Linkedin}
\EndCourses
\switchcolumn
\begin{tabular}{p{3cm}c}
\tskl{Cloud Admin}{3}
\tskl{AWS Cloud}{4}
\tskl{DevOps}{3}
\end{tabular}
\end{paracol}

\newcommand{\Project}[5]{
\hspace{-1em}\raisebox{\dimexpr\ht\strutbox-\height}{\includegraphics[width=0.12\textwidth]{#1}} & #2 & \textbf{#3} \newline \href{http://#4}{\textcolor{blueGray}{#4}} \newline #5 \\ 
}

\section{Projekte}

\begin{paracol}{2}
\BeginCourses
\Project{arduino_btc_project.jpg}{Nov 2023}{BTC Arduino monitor}{github.com/Fdondi/arduino-btc}{Ein Bot überprüft die Bitcoin-Preise und blinkt mit einer grünen LED, wenn der Preis über dem Durchschnitt liegt, und rot, wenn er darunter liegt. Die Blinkgeschwindigkeit hängt vom Ausmaß der Änderung ab.}

\Project{ai_keyboard.jpg}{Aug 2024}{AI Keyboard}{github.com/Fdondi/ai\_keyboard}{Eine Tastatur, um mit einem Klick verschiedene LLM-Operationen auszuführen. Bluetooth-Verbindung ist in Arbeit. }
\EndCourses
\switchcolumn
\begin{tabular}{p{3cm}c}
\tskl{C++ Embedded}{3}
\tskl{Arduino}{2}
\tskl{SSH embedded}{1}
\tskl{Embedded Development}{2}
\tskl{Electronics}{2}
\tskl{Embedded connectivity (USB, WiFi/SSH, Bluetooth LE)}{1}
\end{tabular}
\end{paracol}

\section{Hobbys und Interessen}

Lesen, Wandern, Radfahren, Schwimmen.

\end{document}
